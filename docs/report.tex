\documentclass[french]{article}
 
\usepackage[utf8]{inputenc}
\usepackage[T1]{fontenc}
\usepackage{babel}
\usepackage{advdate}
\usepackage{graphicx} 
\usepackage{amsmath} 
\usepackage{listings}
\usepackage{hyperref}

\begin{document}
\begin{titlepage}
\SetDate[29/04/2018]
\newcommand{\HRule}{\rule{\linewidth}{0.5mm}}
\center 
\textsc{\LARGE Université Paris Dauphine}\\[1.5cm] 
\textsc{\Large Data Analytics}\\[0.5cm]
\HRule \\[0.4cm] { \huge \bfseries
Implementation de KMeans avec Spark}\\[0.4cm] \HRule \\[1.5cm]
\begin{minipage}{0.4\textwidth}
	\begin{flushleft} \large
		\emph{Etudiants}
		\\ Elie \textsc{Abi Hanna Daher}
		\\ Bilal \textsc{El Chami}
		\\ Badr \textsc{Erraji}
	\end{flushleft}
\end{minipage}
~
\begin{minipage}{0.4\textwidth}
	\begin{flushright} \large
		\emph{Professeur} 
		\\ M. Benjamin \textsc{Negrevergne}
		\\  \hspace{1cm}
		\\  \hspace{1cm}
	\end{flushright}
\end{minipage}\\[2cm]
{\large \today}\\[2cm]
\includegraphics[width=8cm]{img/dauphin44e.png}
\vfill
\end{titlepage}
 
\tableofcontents 
\newpage

\section{But du projet}
Le but du projet est d'implémenté l'algorithme KMean avec Spark (python) en évaluant la performance de notre implémentation basé sur des données que nous générons.

\section{Implémentation}

\section{Générateur des données}

\section{Performance}

\end{document}